\right{\1{Section 4}}
\2{high availability and scalibility} \{
  \> scalibility - means that an application/system can handle greater loads by adapting \{
    \> vertical scalibility - increasing the size of the instance
    \> horizontal scalibility (= elasticity) - increasing the number of instances/systems
  \}
  \> high availability - means running your application/system across multi AZ \{
    \< goal is to survive a data center loss
  \}
  \<
\}

\2{Load Balancing}\{
  \> \b{load balancers} are servers that forward internet traffic to multiple servers (EC2 instances) downstream
  \> why use a load balancer \{
    \> spread load across multiple downstream instances
    \> expose a single point of access (DNS) to your application
    \> seamlessly handle failures of downstream instances
    \> health checks
    \> provdie SSL termaination (HTTPS) to your websites
    \> high availability accross zones
    \> separate public traffic from private traffic
    \<
  \}
  \> ELB (EC2 Load Balancer) - managed load balancer \{
    \> AWS guarentees that it will be working, takes care of upgrades, maintenance, high availability
    \> intergrated with many AWS offerings/services
    \> It costs less to setup your own load balancer but it will be a lot more effort on your end 
    \<
  \}
  \> Health Checks - are crucial for load balancers \{
    \> enable the load balancer to know if instances it forwards traffic to are available to reply to requests 
    \> the health check is done on a port and a route (/health is common)
    \> if the response is not 200, then the instance is unhealthy
    \<
  \}
  \> Types of Load Balancers on AWS \{
    \> Classic Load Balancer (CLB) (old) - HTTP , HTTPS, TCP
    \> Application Load Balancer - HTTP, HTTPS, Websocket
    \> Network Load Balancer - TCP, TLS (secure TCP), UDP
    \< 
    \> you can setup internal (private) or external (public) ELBs
    \<
  \}
  \> Good to Know \{
    \> LBs can scale but not instantaneously
    \> Troubleshooting \{
      \> 4xx errors are client induced errors
      \> 5xx errors are application induced errors
      \> Load balancer errors 503 means at capacity or no registered target
      \> If the LB can't connect to your application, check your security groups
    \}
    \> Monitoring \{
      \> ELB access logs will log all access request
      \> CloudWatch Metrics will give you aggregate statistics
      \<
    \}
  \}
  \> Classic Load Balancers \{
    \> Supports TCP (Layer 4), HTTP (Layer 7) and HTTPS (Layer 7)
    \> Health checks are TCP or HTTP based
    \> fixed hostname (xxx.region.elb.amazonaws.com)
  \}
\}
